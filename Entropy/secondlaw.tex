\subsubsection{\texorpdfstring{The 2\textsuperscript{nd} Law of Thermodynamics}{The 2nd Law of Thermodynamics}}
Now that we've gotten through all of that, we can finally talk about the second law of thermodynamics. It states that:
\begin{center}
    \textbf{The entropy of an isolated system cannot decrease, and will increase if possible.}
\end{center}
Or, put another way, an isolated system will always progress towards a state of maximum entropy. Say you have an isolated system with a few closed systems inside of it. It just so happens that the maximum entropy of the entire isolated system is reached when all the closed systems inside of it are at the same temperature! (The exact reason of why this is the maximum entropy state will be revealed very soon!) This gives us a reason why heat flows from hot to cold, to maximize entropy.
\newline\newline
I'm going to backtrack a bit here to make a very important point. The proper definition of a reversible process is based on the second law of thermodynamics. A reversible process is one in which the entropy of the isolated system does not change. Since entropy is a function of state, running this process in reverse would also result in no entropy change. So, the second law of thermodynamics allows the process to run in both directions. If the process instead saw the entropy of the system increase, running it in reverse would see the entropy of the system decrease, which violates the second law of thermodynamics. The only truly isolated system is the entire universe\footnote{Depending on who you ask, maybe not.}, so a process is reversible if and only if it does not change the entropy of the universe.
\newline\newline
Using this idea of reversibility, we can see that no macroscopic process be truly reversible. For an isothermal process we can say $\Delta S \approx \frac{Q}{T}$\footnote{Ask the chemists!}, which clearly isn't zero unless nothing happens at all. Hence, it's not reversible in an isolated system.
\newline\newline
However, we still have some gaps in our understanding regarding entropy. For example, if right now I gave you the current state of a system, you couldn't calculate it's entropy! We've only characterized the \textit{change} in entropy. That seems silly, so lets fix that.
