\subsection{Practice Problems}
\begin{enumerate}
    \item Do particles in an ideal gas undergo inelastic or elastic collisions, and why? 
    \item \begin{enumerate}
        \item Using the relation $k_{b}N_A=R$ (where $R$ is the gas constant and $N_A$ is Avogadro's number, where $N_A = 6.02 \times 10^{23}$ is the number of molecules in a mole), show that $PV={k_{b}}NT$ is equivalent to $PV=nRT$.
        \item Using the relation from the previous part, along with $E_{th} = \frac{\chi}{2}Nk_bT$, show that $E_{th}=\frac{\chi}{2}nRT$.
        \item Express $E_{th}$ in terms of $\chi$, $P$ and $V$.
    \end{enumerate}
    \item If two systems have the same thermal energy, are they at the same temperature? Why?
    \item Imagine Raja is sitting perfectly still in the Science One Box\textsuperscript{TM}, a space specially designed for quiet study. To ensure complete silence, the space inside of the box is kept as a vacuum (don't worry there's a breathing tube). Is this box in equilibrium with its surroundings?
    \item Consider a very normal box\textsuperscript{TM} of very normal ideal gas\textsuperscript{TM}. What would happen to the gas if...
    \begin{enumerate}
        \item The volume of the box decreased?
        \item The volume of the box increased?
        \item The box sprang a leak?
    \end{enumerate}
    \item Imagine a box of particles that are the mass of IKB, which by some miracle don't collapse into black holes and are in fact completely stable. What would happen to the temperature of the gas if you increased the volume of the box, and why (Hint: Think about the gravitational potential energy, and not the ideal gas law)?
    \item Now, imagine a box filled with a gas composed entirely of electrons. When the volume of the box is decreased, what happens to the temperature of the gas, and why (Hint: This time, think about the electric potential energy)?
    \item Suppose you had two boxes of identical volume, with the particles in each travelling at the same average speed. How would the pressure and temperature of the boxes compare if
    \begin{enumerate}
        \item One box had more particles than the other?
        \item One box had heavier particles than the other?
    \end{enumerate}
    \item How much more energy would I have to supply to a triatomic molecule compared to a monoatomic molecule to get them to be the same temperature? Why?
    \item How many degrees of freedom would a monatomic molecule have in 10 dimensional space\footnote{Hurray for string theory!}? If I were to give the same amount of energy to 3 dimensional and 10 dimensional helium, which would have a greater change in temperature?


\end{enumerate}