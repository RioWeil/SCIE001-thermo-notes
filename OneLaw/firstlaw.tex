\subsection{\texorpdfstring{The 1\textsuperscript{st} Law of Thermodynamics}{The 1st Law of Thermodynamics}}
Let $\Delta E$ be the change in energy of a system, $Q$ be the heat that flows in/out the system, and W the work done on the system. The first law of thermodynamics simply states that the total change in energy of a system is equal to the sum of its mechanical and non-mechanical transfers of energy, or
\begin{equation}
    \Delta E=Q+W
\end{equation}
Be careful with the signs when using this equation! For example, had I defined $W$ as the amount of work done \textit{by} the system, this equation would instead be
\begin{equation*}
    \Delta E=Q-W
\end{equation*}
It is worth noting here that while $W$ and $Q$ are both \textbf{functions of path}; how much work I do or a system does, or how much heat flows in/out of a system depends on how a process happens. However, internal energy is a \textbf{function of state}; it depends only on the current state of the system, but not how the system got there. The moral is that while how a system might gain or lose energy is a function of the path, the internal energy itself is a function of the state of the system only.