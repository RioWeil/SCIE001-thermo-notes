\subsection{Practice Problems}
\begin{enumerate}
\item If we think back to the ideal gas law, a term that we could add to make it more general (applicable to non-ideal gases) is to add a term correcting for molecular interactions. Doing so, we get the equation:
\begin{equation}
    \label{eqn:(33)}
    \left(P+a\frac{n^2}{V^2}\right)V = nRT
\end{equation}  
where $a\frac{n^2}{V^2}$ is the interaction term. Note that if we also add an $-nb$ term to the volume to account for the size of the molecules, we end up with the very general Van der Waals equation:
\begin{equation}
    \left(P+a \frac{n^{2}}{V^{2}}\right)\left(V-n b\right)=n R T
\end{equation}
But for the purposes of this question, let's assume $b=0$ to make our lives easier.
\begin{enumerate}
    \item Manipulate the equation \ref{eqn:(33)} to obtain pressure as a function of temperature, volume, and amount of gas.
    \item Suppose that I isothermally compress the gas given in equation \ref{eqn:(33)} from $V_1$ to $V_2$. How much work do I do on the gas?
\end{enumerate}

\item For each of the 4 (8) processes of isochoric heating, isochoric cooling, isobaric expansion, isobaric compression, isothermal expansion, isothermal compression, adiabatic expansion, and adiabatic compression, consider:
    \begin{enumerate}
        \item Is work done on/done by the gas, or is it zero?
        \item Does heat flow into or out of the gas, or is it zero?
        \item Does the temperature of the gas increase, decrease, or stay the same?
    \end{enumerate}
\item Consider the following heat engine, composed of 10 moles of monoatomic gas, below:
\begin{center}
    \begin{tikzpicture}
    \begin{axis}[
        axis x line=bottom,
        axis y line=left,
        xmin=0, xmax=10,
        ymin=0, ymax=10,
%        % (made labels more common)
%        % (because of the "sketch" type of the plot these should not be needed)
%        xlabel={Volume $(\mathrm{m}^3)$},
%        ylabel={Pressure (Pa)},
        % (changed ticks + labels to normal ticks instead of extra ticks)
        xtick={3,6},
        xticklabels={$2$,$4$},
        ytick={1.5,6},
        yticklabels={$5$,$15$},  % <-- (changed order of entries)
    ]
        % fill the area below the curve
        % (draw it first, so it is below everything else)


        % draw the dashed lines
        % (using two different approaches)
        \addplot [dashed,domain=0:3,samples=2] {6};
        \addplot [dashed,domain=0:3,samples=2] {1.5};

        \draw [dashed,thin] (axis cs:6,1.5) -- (axis cs:6,0);
        \draw [dashed,thin] (axis cs:3,1.5)   -- (axis cs:3,0);

        % now draw the curve
        \draw [
            fleche={0.55:black}              % <-- added
        ] (axis cs:3,6) to
            % store start and end coordinates
            coordinate [pos=1] (start)
            coordinate [pos=0] (end)
        (axis cs:6,1.5);
        \draw[fleche={0.55:black}] (axis cs:6,1.5) to (axis cs:3,1.5);
        \draw[fleche={0.5:black}] (axis cs:3,1.5) to (axis cs:3,6);

        % draw start and end point
        \fill [radius=2pt]
            (start) circle[]
            (end)   circle[];
        \filldraw[fill=black] (axis cs: 3,1.5) circle (2pt);
    \end{axis}
        \node[below] at (6.5,0) {$V (m^3)$};
        \node[left] at (0,5.5) {$P (kPa)$};
        \node[above] at (2.05,3.5) {$3$};
        \node[right] at (4.25,0.9) {$1$};
        \node[left] at (2.1,1.15) {$2$};
\end{tikzpicture}
\end{center}
\begin{enumerate}
    \item What is the net change in the internal energy of the heat engine after one cycle?
    \item What is the work done by the heat engine in one cycle? (in kiloJoules... note that $1\textrm{Pa}*1\textrm{m}^3 = 1\textrm{J}$)
    \item What is the net heat flow of the heat engine in one cycle? (in kiloJoules)
    \item Solve for the temperature (in Kelvin) at each of the three points.
    \item For each step, determine the work done by the engine, and how much heat flows into/out of the engine. 
\end{enumerate}
\item Consider the following heat engine, composed of monoatomic gas. Treat the process $3 \rightarrow 1$ as isothermal.
\begin{center}
    \begin{tikzpicture}
    \begin{axis}[
        axis x line=bottom,
        axis y line=left,
        xmin=0, xmax=10,
        ymin=0, ymax=10,
%        % (made labels more common)
%        % (because of the "sketch" type of the plot these should not be needed)
%        xlabel={Volume $(\mathrm{m}^3)$},
%        ylabel={Pressure (Pa)},
        % (changed ticks + labels to normal ticks instead of extra ticks)
        xtick={3,6},
        xticklabels={$V_0/2$,$V_0$},
        ytick={1.5},
        yticklabels={$P_0$},  % <-- (changed order of entries)
    ]
        % fill the area below the curve
        % (draw it first, so it is below everything else)


        % draw the dashed lines
        % (using two different approaches)
        \addplot [dashed,domain=0:3,samples=2] {1.5};

        \draw [dashed,thin] (axis cs:6,1.5) -- (axis cs:6,0);
        \draw [dashed,thin] (axis cs:3,1.5)   -- (axis cs:3,0);

        % now draw the curve
        \draw [
            fleche={0.55:black}              % <-- added
        ] (axis cs:3,6) to [bend right = 30]
            % store start and end coordinates
            coordinate [pos=1] (start)
            coordinate [pos=0] (end)
        (axis cs:6,1.5);
        \draw[fleche={0.55:black}] (axis cs:6,1.5) to (axis cs:3,1.5);
        \draw[fleche={0.5:black}] (axis cs:3,1.5) to (axis cs:3,6);

        % draw start and end point
        \fill [radius=2pt]
            (start) circle[]
            (end)   circle[];
        \filldraw[fill=black] (axis cs: 3,1.5) circle (2pt);
    \end{axis}
        \node[below] at (6.5,0) {$V$};
        \node[left] at (0,5.5) {$P$};
        \node[above] at (2.05,3.5) {$3$};
        \node[right] at (4.25,0.9) {$1$};
        \node[left] at (2.1,1.15) {$2$};
\end{tikzpicture}
\begin{enumerate}
    \item What is the ratio of the temperatures between point $3$ and point $1$?
    \item Solve for the pressure at point $3$ in terms of $P_0$ and $V_0$
    \item Mark the highest temperature $T_H$ and the lowest temperature $T_C$ on the graph. 
    \item Solve for the efficiency of the heat engine. Your answer should not depend on $P_0$, $V_0$, or the amount of gas in the heat engine. (Note: Solving for the efficiency is quite a bit more difficult than the other parts; a partial solution of the work and heat done during certain steps would be a good partial solution.)
\end{enumerate}
\end{center}
\item Question 4 again, except this time treat process $3 \rightarrow 1$ as adiabatic (Note that part (d) is \textit{quite} difficult).
\item Use what you know of adiabatic and isothermal processes to finish deriving the Carnot efficiency.

\end{enumerate}