\subsubsection{Isothermal Processes}
Isothermal processes are compression/expansion processes in which the temperature of the gas remains constant ($\Delta T = 0$). These processes occur very slowly. As an example, we consider a scenario where I have some gas in a cylinder (closed off at the top by a piston), where I push down on the piston very very slowly. Doing so, the gas inside the piston remains at thermal equilibrium with its surroundings. Hence, I as I do work on the gas, an equal amount of energy leaves the gas in the form of heat, and the gas inside the piston stays at the same temperature. \\
\noindent
One immediate consequence that we obtain from $\Delta E = nc_v\Delta T$ is that:
\begin{equation}
    \Delta E = 0
\end{equation}
And by the first law of thermodynamics, we have that:
\begin{equation}
    W = -Q
\end{equation}
What exactly is the amount of work done in an isothermal process? We once again return to the definition of work:
\[ W = -\int_{V_1}^{V_2}P(V)dV \]
We have to be very careful here though; in this situation, the pressure is not constant as we vary/integrate over the volume. So we have to be a bit clever, and think of a way in which we can express the pressure as a function of volume so we can carry out the integral. If your intution was "ideal gas law", then you'd be exactly right! Let us make the substition:
\[ P(V) = \frac{nRT}{V} \]
Which turns the integral into:
\[ W = -\int_{V_1}^{V_2}nRT\frac{dV}{V}\]
Throughout this process $T$ is held constant, and the amount of gas does not change, so we can take most of the terms out of the integral as a constant:
\[ W = -nRT\int_{V_1}^{V_2}\frac{dV}{V}\]
The remaining integral is straightforward:
\[ W = -nRT \left. \ln(V) \right|_{V_1}^{V_2} = -nRT\left(\ln(V_2)-\ln(V_1)\right) = nRT\left(\ln(V_1)-\ln(V_2)\right) \]
So applying some laws of logarithms, we hence obtain our expression for the work done on the gas in an isothermal process:
\begin{equation}
W = nRT\ln\left(\frac{V_1}{V_2}\right)
\end{equation}
As we found earlier, the heat flow during an isothermal process is simply the negative of this, so:
\begin{equation}
    Q = -nRT\ln\left(\frac{V_1}{V_2}\right) = nRT\ln\left(\frac{V_2}{V_1}\right)
\end{equation}
\noindent
Finally, we consider how isothermal curves show up on PV diagrams. From how the expression for the work looks (with the $\ln$ term), you may suspect that we might get a "curved" curve rather than a straight diagonal line, and you would be right. Pictured here is an isothermal expansion from initial state $P_1,V_1$ to final state $P_2,V_2$. The temperature stays constant at every point on the curve. 

\begin{center}
    \begin{tikzpicture}
    \begin{axis}[
        axis x line=bottom,
        axis y line=left,
        xmin=0, xmax=10,
        ymin=0, ymax=10,
%        % (made labels more common)
%        % (because of the "sketch" type of the plot these should not be needed)
%        xlabel={Volume $(\mathrm{m}^3)$},
%        ylabel={Pressure (Pa)},
        % (changed ticks + labels to normal ticks instead of extra ticks)
        xtick={3,6},
        xticklabels={$V_1$,$V_2$},
        ytick={1.5,6},
        yticklabels={$P_2$,$P_1$},  % <-- (changed order of entries)
    ]
        % fill the area below the curve
        % (draw it first, so it is below everything else)


        % draw the dashed lines
        % (using two different approaches)
        \addplot [dashed,domain=0:3,samples=2] {6};
        \addplot [dashed,domain=0:6,samples=2] {1.5};

        \draw [dashed,thin] (axis cs:6,1.5) -- (axis cs:6,0);
        \draw [dashed,thin] (axis cs:3,6)   -- (axis cs:3,0);

        % now draw the curve
        \draw [
            fleche={0.6:black},              % <-- added
        ] (axis cs:3,6) to [bend right=30]
            % store start and end coordinates
            coordinate [pos=1] (start)
            coordinate [pos=0] (end)
        (axis cs:6,1.5);

        % draw start and end point
        \fill [radius=2pt]
            (start) circle[]
            (end)   circle[];
    \end{axis}
        \node[below] at (6.5,0) {$V$};
        \node[left] at (0,5.5) {$P$};
        \node[right] at (2.2,3.4) {$T_1$};
        \node[right] at (4.25,0.9) {$T_1$};
\end{tikzpicture}
\end{center}
